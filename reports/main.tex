% Took this from CogSci but removed the header, sorry.
%
% Author: Matthew Turner
% Date: 2017-11-23

\documentclass[11pt,letterpaper]{article}

\usepackage{lineno}
\linenumbers
% \usepackage{cogsci}
\usepackage{fullpage}
\usepackage{booktabs}
\usepackage{setspace}
\doublespacing
\usepackage{pslatex}
\usepackage{hyperref}
\usepackage{url}
\usepackage{apacite}
\usepackage{amsmath}
\usepackage{subcaption}
\usepackage[utf8]{inputenc}
\usepackage{pgfplots}
\pgfplotsset{compat=newest}
\usepgfplotslibrary{groupplots}
\usepackage{wrapfig}
\usepackage{bigfoot}
\usepackage[export]{adjustbox}
\setlength\intextsep{0pt}
\usepackage{authblk}



\usepackage{graphicx}

\usepackage{gb4e}  % linguistic examples
\noautomath

\title{Polarization is sensitive to initial opinion extremity, 
agent update path, and communication noise}

\author[1]{Matthew~A.~Turner}
\author[1]{Paul~E.~Smaldino}

\affil[1]{\footnotesize Cognitive Science Program, University of California, Merced}

\date{}

\begin{document}
\maketitle

% \begin{abstract}
%   Opinion dynamics in general models how attributes of agents in a population
%   change over time. This paper reproduces and extends the investigation of
%   \citeA{Flache2011} into how cultural polarization emerges in populations
%   connected on a small-world network. Their results do much to shed light
%   on what might be the necessary requirements for polarization to occur.
%   Importantly, opinions must be measured in a way that allows for negative
%   valence of opinion, representing being ``against'' some cultural object.
%   Here we add three further considerations we believe are important in 
%   modeling when polarization obtains. First, we show that final polarization
%   is sensitive to the initial extremity of agent opinions. Second, we quantify
%   how robust polarizing dynamics are to noise. 
% \end{abstract}

\section{Introduction}

\subsection{Key results}

\begin{itemize}
  \item \textbf{Final polarization is update path-dependent.}
    \begin{itemize}
      \item (Figure~\ref{fig:single-experiments-over-k})
        Final polarizations can obtain for the non-random network that
        are larger than the average of trials of the small world network
        generated by adding random long-range ties. In other words, a
        small world network is not necessary for large polarization values.
        It is also not sufficient, as small-world networks can collapse to
        values of polarization smaller than the average for the connected
        caveman network trials. 
      \item (Figure~\ref{fig:final-initial-pol-regplot}) In the non-random
        condition, final polarization is correlated wtih initial polarization,
        but not perfectly ($r=0.37$). Since the network structure is uniform in this
        experiment, the rest of the variance must come from differences in 
        update paths.
      \item (Figure~\ref{fig:highpol-histogram}) 
        Starting with identical initial distributions
        and the non-random connected caveman network results in a distribution
        of final polarizations. This demonstrates that, for the parameters
        tested, initial conditions in general do not determine final system
        configuration. It seems that initial conditions \emph{bias} the system
        towards higher final polarizations.
    \end{itemize}
  \item \textbf{Final polarization is sensitive to initial conditions.}    
    \begin{itemize}
      \item (Figure~\ref{fig:final-initial-pol-regplot}) For the connected 
        caveman graph and the default Flache and Macy initial conditions ($S=1.0$),
        final polarization and initial polarization are 
        correlated ($r=.37$). 
      \item (Figures~\ref{fig:zoom_average},~\ref{fig:zoom_median}) 
        As we artificially lowered the magnitude of 
        initial agent opinions, large final polarizations became more rare
        in the small world experiment of \citeA{Flache2011}: 
        20 long-range random ties added at iteration 2000.
      \item (Figure~\ref{fig:single_S_K}) As we showed for the connected 
        caveman case, initial conditions do not completely determine the
        final polarization. As we decreased $S$, polarization did not 
        decrease completely even when average and median polarization vanished.
        Some simulations still ended in highly polarized final states.
    \end{itemize}
  \item \textbf{Final polarization is sensitive to communication noise.}    
    \begin{itemize}
      \item (Figure~\ref{fig:heatmaps})
      \item (Figure~\ref{fig:single-runs-commnoise})
    \end{itemize}
\end{itemize}

\subsection{The rest of the intro}

In the most abstract, at any point in time, culture is the state of all
constituent minds each in their individual states. Culture changes when 
individuals change. Individuals change under the influence of other 
individuals. How do individuals change their mental state and how is the
state of culture measured as a whole? There are many possible ways. A popular
metric for characterizing the joint mental state 
of a population is \emph{polarization}.
Polarization can be measured a multitude of ways \cite{Bramson2016}. The 
concept of polarization has meaning in physics as well---the water molecule
is slightly polarized, for example, giving rise to the van der Waal's force
and high surface tension. In this paper we consider the structural forces that
arise and we call ``polarization'' with respect to agent mental states called
opinions. 

One of the first was \cite{Axelrod1997}, where each cultural feature was a 
digit from 1-9. 

In the model we consider, due to \citeA{Flache2011}, agents' opinions
change based on a weighted difference of their opinions with their neighbors
opinions. Agents are nodes on a network, and relationships are edges of the
network. The edges of this model are weighted and undirected, so each 
connected pair of agents has equal influence over one another, as far as
weights are concerned. A feature of this model is that extremists' opinions
are more entrenched than centrists' opinions, so if an extremist and centrist
are connected, the extremist in a dyad has more \emph{effective} influence 
than a centrist. 

Each agent's mental state is a vector of opinions on a set of cultural features.
The number of cultural features in the model is called the \emph{cultural 
complexity}. We adopt the viewpoint that cultural evolution is a form
of distributed computation \cite{Smaldino2013}. What exactly does the system
compute? It computes its own equilibrium states. We show in this paper that
just what equilibrium state the system finds depends not only on initial
conditions, but also on the order agent opinions and inter-agent weights are 
updated. The results from \citeA{Flache2011} might suggest that maximum 
levels of system polarization only obtain if the cultural complexity is 
sufficiently small, or if the network is in the ``small world'' condition
induced by randomizing a connected caveman network. To be fair, \citeA{Flache2011}
do not make this strong of a conclusion, but they also do not explore the
sources of variation of the final polarization of individual model runs 
that are aggregated to mean values and presented in that paper.
We demonstrate here that particular values of 
initial conditions, cultural complexity, and 
network structure are not necessary conditions for high levels of 
final polarization. Instead, these factors should be understood as biases 
that favor evolution towards equilibria of higher or lower final polarization. 
Specifically, we demonstrate that non-randomized connected caveman 
networks can also reach near-maximal final polarization 
equilibria, as can systems with large cultural
complexities. We demonstrate that there is a smallest initial condition that
causes all system evolutions to reach consensus for any cultural complexity
that is greater than a trivial, obvious one. In doing so, we show that the
update path can lead a system with identical initial conditions to different 
final polarizations. We support our claim that the system is finding local 
extrema by demonstrating that there is a critical level of communication noise
that will percolate the system to steady states of maximal polarization. 
This is similar to simulated annealing, but the system freezes itself 
due to the fact that extreme positions become increasingly immobile in 
the \citeA{Flache2011} model.

\subsection{Summary of Flache and Macy (2011) model}

\section{Methods}
\label{sec:methods}

\section{Results}

\subsection{Variation in original Flache and Macy Results}

We first explore variability in the original \citeA{Flache2011} results.
We show below that there is in fact large variability in final polarization
values that aggregate to the means presented in that paper. What causes
this variability? 

One main result of
\citeA{Flache2011} was that polarization decreases with cultural complexity.
To demonstrate our model is working as theirs is, compare Figure 
\ref{fig:p_vs_K_fm} to Figure 12b of \citeA{Flache2011}---these figures
are identical.
However, when plotting the mean as well as the results of individual trials,
we see that the mean captures some, but not all, of the detail of what's 
happening in terms of opinion dynamics. It is not at all rare for trials to
have final polarizations far from the average 
(Figure~\ref{fig:p_vs_K_single_experiments}). 


Where does
this variation come from? It seems there are two sources: the specific
initial opinions that agents draw from the uniform distribution and the
update path. As (WILL BE) explained in the introduction/methods, 
the \citeA{Flache2011} an iteration is defined as $N$ updates of either a 
weight or an edge.  By ``update path,'' we mean which edges or weights are 
updated in what order. 

First, to see what effect the exact initial distribution has on the final 
polarization we regressed final polarization against initial polarization for
$K=2$ for the non-random connected caveman graph 
(Figure~\ref{fig:final-initial-pol-regplot}, whose final polarizations are 
plotted in the $K=2$ column of Figure~\ref{fig:connected-caveman-trials}.
While there is a trend, there is still much noise with final polarizations
far above and below the regression line. 

To evaluate the path dependence, we ran another 100 trials using the initial
conditions from the $K=2$, connected caveman trial
with the maximum final polarization of 0.86. The distribution of outcomes was
skewed towards polarizations higher than the mean for connected caveman, $K=2$,
which was .41 (Figure~\ref{fig:highpol-histogram}). However, there are many,
though a minority, of trials that end \emph{below} that mean. This seems to
suggest that update path has a large impact on the final polarization of a
given trial. However, initial conditions can bias the final polarization. 

Why is this? Consider the space of all update paths. 
There are ($N_{iter} \times N_{agents} \times N_{weights}$?) of these.
I hypothesize this is because larger initial polarizations means a larger
fraction of these paths are polarizing. This may well be related to the 
concept of signed graphs: graphs where one can perfectly separate the graph
into two graphs where all agents are friendly with each other, i.e. 
$w_{ij} > 0$ for all agents $i,j$ \cite{Altafini2012}. There are measures of
signedness, and I have a feeling that signedness tracks with polarization.
That is, as the iterations progress, we could plot signedness as well as 
polarization, and when signedness breaks different thresholds, higher levels of
polarization become inevitable.


\subsection{Initial opinion extremism and polarization}

\citeA{Flache2011} considered the effect of modifying two parameters under their
``Experiment 2.'' This experiment, which is the only one we consider in this
work, experimented with different values of cultural complexity, $K$, and 
different cave sizes. We were curious to understand more about exactly when
polarization emerges, especially with respect to the extremity of opinion
for different cultural complexities. To quantify the affect of extremity of
opinion, we introduced a new parameter $S \in [0, 1]$ that constrains the
extremity of initial opinions. Instead of initial opinions on each
cultural feature being drawn with uniform probability from $[-1, 1]$ they are
instead drawn from $[-S, S]$. 

We performed copmutational experiments to determine the minimum $S$ 
in order for non-zero (greater than 1e-6) polarization to emerge. Clearly if 
$S\leq0.5$ there can be no polarization because all distance are less than 1,
which means all weights will be positive, all
relationships will be friendly, and all agents will be drawn towards some 
consensus point. Our initial experiments (Figure \ref{fig:p_vs_s_for_k})
showed that average polarization was zero for $S < 0.75$ when $K=2$. 
The critical $S$ seems to change for different values of $K$. For example,
average polarization does not increase until $S=0.8$ for $K=3$. For other $K$,
polarization seems to increase to different degrees, with some noise, starting
between $S=0.85$ and $S=0.9$. It appears for $K=3$ that average polarization
decreases from $S=0.95$ to $S=1.0$. There is a large bump in average 
polarization at $S=0.9$ for $K=6$. Is this something real due to some special
configuration for these parameters, or is it noise? (Of course it is not a bug
in our code!)

To better understand the behavior we saw there, we ran a new set of simulations
with 100 trials for each parameter setting, and ran $S$ from 0.75 to 1.0 for
$K=2, 3, 4$ and from 0.85 to 1.0 for $K=5,6$, all with a smaller step size 
of 0.01 (Figure~\ref{fig:zoom_average}). Since the results still appear
noisy, we also plotted median polarization verus $S$ for the different $K$
(Figure~\ref{fig:zoom_median}). The median is still noisy. So, to get even
more detail, we visualized each trial result, and the average, individually
over the $S$ values we tested with 100 trials (Figure~\ref{fig:single_S_K}).
There is indeed a large amount of variation, which seems to increase as $S$
increases.


% \begin{figure}[t!]
%   \centering
%     \includegraphics[width=0.75\textwidth]{Figures/p_v_s_k6_histograms.pdf}
%   \caption{Distribution of non-zero final polarization values for $K=6$.
%     Corresponds to the experiment run shown in Figure \ref{fig:zoomK6}.
%   }
%   \label{fig:K6-histograms}
% \end{figure}


\subsection{Sources of single-trial polarization}


\subsection{Communication noise and polarization}



\section{Discussion}

Our results suggest that, in real-world political problems, 
like the previously-mentioned constitutional crises in Egypt or Venezuela,
it is crucial to change the opinions of certain actors before others. Such 
situations are not fated to descend into chaos and violence. It is not a
random number generator that determines who we interact with and learn from 
in the real world. We do not have to choose the path that leads to higher 
polarization, we can choose one that leads to an equilibrium with a lower
polarization.  We have shown that understanding one another well can lead to 
less polarized outcomes. If we do not, it does not matter how politically savvy
a society is, it will find itself in a state of high tension.

Our results suggest deeper principles at work, and we now must ask, how can
we characterize the different update paths?

\clearpage

\begin{figure}
  \centering
    \includegraphics[width=0.75\textwidth]{Figures/p_vs_K_fm.pdf}
  \caption{Reproduction of Figure 12b of Flache and Macy (2011). The average
    polarization decreases with $K$. However, as shown in subsequent figures,
    this does not mean trials with high polarization never obtain. Average
    taken over 100 trials.
  }
  \label{fig:p_vs_K_fm}
\end{figure}


\begin{figure}[h!]
  \centering
    \begin{subfigure}[t]{\textwidth}
      \centering
      \includegraphics[width=.55\textwidth]{Figures/connected-caveman-over-K.pdf}
      \caption{Non-random connected caveman network.}
      \label{fig:connected-caveman-trials}
    \end{subfigure}
    \begin{subfigure}[t]{\textwidth}
      \centering
      \includegraphics[width=.55\textwidth]{Figures/random-shortrange-over-K.pdf}
      \caption{Random connected caveman network with short-range random ties added at iteration 2000.}
      \label{fig:random-shortrange-trials}
    \end{subfigure}
    \begin{subfigure}[t]{\textwidth}
      \centering
      \includegraphics[width=.55\textwidth]{Figures/random-anyrange-over-K.pdf}
      \caption{Randomized connected caveman network with long-range random ties added at iteration 2000.}
      \label{fig:random-anyrange-trials}
    \end{subfigure}
  \caption{Results of indivdual model runs for different network conditions. 
    The averages of these were shown in Figure~\ref{fig:p_vs_K_fm}.
    Even in the non-random connected caveman structure, there is 
    variation in the final polarization for different values of $K$. Highly
    polarized final states may obtain. 100 trials are shown for each 
    network condition and comprise the
    average. The circled square in~\ref{fig:connected-caveman-trials}
    highlights the experimental configuration
    reused to make Figure~\ref{fig:highpol-histogram}.
  }
  \label{fig:single-experiments-over-k}
\end{figure}


\begin{figure}[h!]
  \centering
    \includegraphics[width=.75\textwidth]{Figures/final-initial-pol-regplot.pdf}
  \caption{Regression of final polarization against initial polarization 
    for $K=2$ in the non-random connected caveman network configuration.
    Final polarizations are same as in the $K=2$ column of 
    Figure~\ref{fig:connected-caveman-trials}. 100 trials are shown. The
    top histogram shows the distribution of initial polarization across
    trials. The right histogram shows the distribution of final polarization
    across trials.}
  \label{fig:final-initial-pol-regplot}
\end{figure}

\begin{figure}[h!]
  \centering
    \includegraphics[width=.75\textwidth]{Figures/caveman-highpol-histogram-10k-its.pdf}
  \caption{Distribution of final polarizations at iteration 10000
  starting from initial conditions of the connected caveman trial with 
  maximal final polarization for $K=2$ shown in 
  Figure~\ref{fig:connected-caveman-trials}.
  The distribution is skewed towards final polarizations considerably larger
  than the mean polarization of 0.41 for the connected caveman experiment
  with $K=2$ found by Flache and Macy and shown in Figure~\ref{fig:p_vs_K_fm}. 
  }
  \label{fig:highpol-histogram}
\end{figure}

% \begin{figure}[t!]
%   \centering
%   \includegraphics[width=0.75\textwidth]{Figures/P_vs_S_for_K.pdf}
%   \caption{
%     Average final polarization becomes non-zero then increases as
%     the width of the uniform distribution of initial opinions increases.
%     The width must be larger and larger as the cultural complexity $K$ 
%     increases for the system to achieve non-zero final polarization. This
%     condition is identical to the bottom row of each heatmap in 
%     Figure \ref{fig:heatmaps}. Each
%     data point is the average of fifty trials. Each trial ran over 
%     10k timesteps. 
%   }
%   \label{fig:p_vs_s_for_k}
% \end{figure}


\begin{figure}[t!]
  \centering
    \includegraphics[width=0.75\textwidth]{Figures/s_k_zoom_2-6_mean.pdf}
  \caption{Average polarization for different cultural complexities over 
    maximum initial opinion magnitude, $S$. 
    Averages are roughly zero for $S<0.75$ for all cultural complexities.
    Each data point is the average over 100 trials of the condition
    where long-range ties were added at iteration 2000. Each trial was run to 
    10k total iterations. 
  }
  \label{fig:S_average}
\end{figure}

\begin{figure}[h!]
  \centering
    \includegraphics[width=0.75\textwidth]{Figures/s_k_zoom_2-6_median.pdf}
  \caption{Median polarization for different cultural complexities over
    maximum initial opinion magnitude, $S$.
    Median polarization for $K=5$ and $K=6$ are both flat at zero; $K=5$ 
    data is obscured by $K=6$.  Median taken over 100 trials.
    Same data as in Figure~\ref{fig:S_average}, so the condition was
    long-range ties added at iteration 2000, with each trial run to 10k
    total iterations.
  }
  \label{fig:S_median}
\end{figure}


\begin{figure}[h!]
  \centering
  \begin{subfigure}[t]{\textwidth}
    \centering
    \includegraphics[width=0.5\textwidth]{Figures/single_S_K=2.pdf}
  \end{subfigure} \\
  \begin{subfigure}[t]{0.49\textwidth}
      \centering
      \includegraphics[width=\textwidth]{Figures/single_S_K=3.pdf}
      % \caption{}
  \end{subfigure}
  ~
  \begin{subfigure}[t]{0.49\textwidth}
      \centering
      \includegraphics[width=\textwidth]{Figures/single_S_K=4.pdf}
      % \caption{}
  \end{subfigure} \\
  \begin{subfigure}[t]{0.49\textwidth}
      \centering
      \includegraphics[width=\textwidth]{Figures/single_S_K=5.pdf}
      % \caption{}
  \end{subfigure}
  ~
  \begin{subfigure}[t]{0.49\textwidth}
      \centering
      \includegraphics[width=\textwidth]{Figures/single_S_K=6.pdf}
      % \caption{}
  \end{subfigure}
  \caption{Final polarization of individual trial runs and averages from
    Figure~\ref{fig:S_average}. Each trial was run in the condition 
    where random long-range ties were added at iteration 2000 and each trial
    was run to 10k iterations.
  }
  \label{fig:single_S_K}
\end{figure}


\begin{figure}
  \centering
    \includegraphics[width=.75\textwidth]{Figures/min_parallel_K=2.pdf}
  \caption{}
  \label{fig:}
\end{figure}

\begin{figure}
  \centering
    \includegraphics[width=.75\textwidth]{Figures/max_parallel_K=2.pdf}
  \caption{}
  \label{fig:}
\end{figure}



\begin{figure}[t!]
  \centering
      \begin{subfigure}[t]{0.49\textwidth}
          \centering
          \includegraphics[width=\textwidth]{Figures/noisecomm_K=2.pdf}
          % \caption{}
      \end{subfigure}
      ~
      \begin{subfigure}[t]{0.49\textwidth}
          \centering
          \includegraphics[width=\textwidth]{Figures/noisecomm_K=3.pdf}
          % \caption{}
      \end{subfigure} \\
      \begin{subfigure}[t]{0.49\textwidth}
          \centering
          \includegraphics[width=\textwidth]{Figures/noisecomm_K=4.pdf}
          % \caption{}
      \end{subfigure}
      ~
      \begin{subfigure}[t]{0.49\textwidth}
          \centering
          \includegraphics[width=\textwidth]{Figures/noisecomm_K=5.pdf}
          % \includegraphics[width=\textwidth]{Figures/p_v_noise_k=5.pdf}
          % \caption{}
      \end{subfigure} \\
  \caption{Final average polarization varies with both the width of the
    uniform distribution of initial opinion magnitudes and the noise level in
    the opinion updates. The value in each square of the heatmap is the average of
    fifty trials. Each trial was run in the condition where random long-range
    ties were added at iteration 2000. Each trial ran to 10k timesteps. 
  }
  \label{fig:heatmaps}
\end{figure}


\begin{figure}[t!]
  \centering
      \begin{subfigure}[t]{0.49\textwidth}
          \centering
          \includegraphics[width=\textwidth]{Figures/noisecomm_S=0p5_K=2.pdf}
          % \caption{}
      \end{subfigure}
      ~
      \begin{subfigure}[t]{0.49\textwidth}
          \centering
          \includegraphics[width=\textwidth]{Figures/noisecomm_level=0p10_K=2.pdf}
          % \caption{}
      \end{subfigure} \\
      \begin{subfigure}[t]{0.49\textwidth}
          \centering
          \includegraphics[width=\textwidth]{Figures/noisecomm_S=0p5_K=4.pdf}
          % \caption{}
      \end{subfigure}
      ~
      \begin{subfigure}[t]{0.49\textwidth}
          \centering
          \includegraphics[width=\textwidth]{Figures/noisecomm_level=0p10_K=4.pdf}
          % \caption{}
      \end{subfigure} \\
      \begin{subfigure}[t]{0.49\textwidth}
          \centering
          \includegraphics[width=\textwidth]{Figures/noisecomm_S=0p5_K=6.pdf}
          % \caption{}
      \end{subfigure}
      ~
      \begin{subfigure}[t]{0.49\textwidth}
          \centering
          \includegraphics[width=\textwidth]{Figures/noisecomm_level=0p10_K=6.pdf}
          % \caption{}
      \end{subfigure}
  \caption{Final polarization of individual trial runs and averages from
    Figure~\ref{fig:heatmaps}. Each trial was run in the condition where
    random long-range ties were added at iteration 2000 and each trial was
    run to 10k iterations. In the left column, the maximum inintial opinion
    magnitude, $S$, is constant at 0.5 and noise level vaies along the x-axis. 
    As the noise level is increased, the
    system is increasingly biased towards larger final polarization outcomes.
    In the right column, the noise level is held constant at 0.1 and $S$ 
    varies along the x-axis. For cultural complexity of $K=6$, most trials do
    not attain large values of final polarization.
  }
  \label{fig:single-runs-commnoise}
\end{figure}


\clearpage

\bibliographystyle{apacite}

\setlength{\bibleftmargin}{.125in}
\setlength{\bibindent}{-\bibleftmargin}

\bibliography{/Users/mt/workspace/papers/library.bib}

\end{document}
